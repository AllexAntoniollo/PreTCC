
\chapter{\label{chap:chap2}Contratos Inteligentes e Privacidade}

Com base em  \cite{GGA}, a consideração sobre modificadores de acesso é frequentemente negligênciada por desenvolvedores de contratos inteligentes. Em contratos inteligentes, o encapsulamento não age como uma medida de segurança efetiva, dado que na blockchain, o acesso à informação é praticamente público para todos os participantes da rede. Embora os modificadores de acesso ajudem a organizar o código e a gerenciar o acesso entre contratos, é crucial compreender que, em blockchains públicas, a visibilidade das informações não é efetivamente controlada por esses modificadores.

O equívoco comum reside na crença de que marcar uma variável como privada a torna segura. No entanto, ao implantar um contrato na blockchain, esse código e todas as suas variáveis associadas são propagados por toda a rede. Isso significa que qualquer indivíduo que possua uma cópia do blockchain terá acesso às informações, independentemente do modificador de acesso.

Em decorrência disso, armazenar dados sigilosos ou delicados em variáveis privadas na blockchain é uma prática arriscada. Torna-se essencial adotar estratégias adicionais, como a criptografia e técnicas de ofuscação, para proteger informações sensíveis, reconhecendo que a simples marcação como "privado" não proporciona o nível de segurança desejado em um ambiente descentralizado e transparente como o das blockchains públicas.


\section{Característica dos Contratos Inteligentes}

Conforme discutido anteriormente, nos sistemas de blockchains públicas tradicionais, a natureza distribuída e pública dessas plataformas pode impactar a integridade e a privacidade dos dados. Essa consideração é relevante também neste contexto.

Em blockchains públicas, a visibilidade generalizada dos dados representa um desafio para a confidencialidade, pois qualquer pessoa pode acessar e verificar todas as transações registradas. No entanto, ao empregar tecnologias avançadas como a Zero-Knowledge Proof (Prova de Conhecimento Zero - ZKP), é possível mitigar esses desafios.

A implementação bem-sucedida de ZKP em sistemas blockchain públicos oferece uma solução eficaz para preservar a privacidade, permitindo que informações sensíveis permaneçam ocultas, mesmo em uma infraestrutura distribuída. Isso não apenas aprimora a segurança, mas também mantém a confiança dos usuários, promovendo a aceitação generalizada dessas plataformas.

 



\subsection{Desafios de Privacidade em Contratos Inteligentes}

\textbf{Prova de Conhecimento zero: }Com os avanços contínuos na tecnologia, a aplicação de ZKP tem se destacado como uma abordagem significativa para aprimorar a privacidade de dados durante transações em blockchain. A ZKP oferece a capacidade de preservar a confidencialidade de detalhes específicos, mesmo quando os dados da transação são registrados na blockchain.

Em ambientes descentralizados, como contratos inteligentes em decentralized applications (dApps), a implementação da prova de conhecimento zero emerge como uma medida crucial para fortalecer a segurança e atratividade dessas aplicações, especialmente aquelas que lidam com informações sensíveis. Ao incorporar ZKP nos contratos inteligentes, é possível assegurar que, mesmo quando os dados da transação são adicionados à blockchain, certos elementos permaneçam estritamente confidenciais.

Essa abordagem não apenas eleva os padrões de privacidade, mas também posiciona as dApps como soluções mais confiáveis para usuários preocupados com a segurança de seus dados. A capacidade de realizar transações sem divulgar informações específicas não apenas atende às demandas de privacidade, mas também impulsiona a confiança dos usuários em ambientes descentralizados \cite{SMPP}.

\textbf{Armazenamento em Banco de Dados: }De fato, a estratégia de armazenar dados sigilosos em um banco de dados, em vez de registrar tudo na blockchain, é uma abordagem alternativa que aborda preocupações de privacidade, mas ao custo da descentralização. Essa consideração também se aplica aqui.

Ao optar por um banco de dados centralizado para armazenar informações confidenciais, a descentralização inerente às blockchains é comprometida. A dependência de um servidor centralizado introduz um ponto único de falha e potenciais vulnerabilidades de segurança, pois um ataque ou falha nesse servidor pode comprometer todos os dados sensíveis.

Embora essa abordagem possa ser eficaz para proteger a privacidade dos dados, ela entra em conflito com os princípios fundamentais da descentralização que muitos sistemas blockchain visam alcançar. A centralização também implica maior controle e responsabilidade nas mãos da entidade que gerencia o servidor, o que pode gerar desconfiança por parte dos usuários.

Ao avaliar as opções entre descentralização e proteção de dados sigilosos, é fundamental encontrar um equilíbrio que atenda aos requisitos específicos da aplicação. 

 
